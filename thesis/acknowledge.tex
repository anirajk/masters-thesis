I consider myself fortunate in more ways than I can count and I thank God profusely for that. 
Among all the people that I could thank for every single achievement in my life, my gratitude for my parents, Kesavan Mundathode Pangottil and Sarala Kesavan, surpasses the rest. 
I have come to realize that the luckiest I have ever been was to be born to my parents who dedicated their entire lives to the betterment of mine in the face of financial hardships and social stigma. 
They instilled in me the value of education early on. No statement of acknowledgment would be complete without thanking them.

I owe everything I have learned and accomplished as a graduate student at the U to Ryan Stutsman. 
Ryan was the rock of my master's education, and if not for his steadfast mentoring and confidence in me despite my inherent skepticism, this work would not have been possible. 
I understand that it is a common sentiment among students to attribute their success to their advisors, but when I say that I could not have done any of this without Ryan, I mean it from the bottom of my heart. 
He has been the greatest mentor, advisor, and instructor and his positivity and unparalleled work ethic inspired me to go the extra mile to complete my tasks. 
He is humble despite his many accomplishments, and he has amazed me with his trust in me during testing times. 
Ryan's empathy and generousness went a long way in my success here. 
I consider myself privileged to be one of the first students to graduate under him and I wish him continued stellar success at the U.

Robert Ricci is the sole reason I decided to attend the U, and no one has been more encouraging to me than Rob during the last two years. 
He showed me the ways of graduate school and was the forcing function that kept me on track of my progress. 
He taught me what to focus on while presenting research and I am a better student for it. 
I could always walk up to his office to get answers to questions on both life and research. 
Since his response to my first email before my application up until my thesis defense, Rob was instrumental in my success here. 
I would also like to thank Feifei Li for sitting on my thesis committee. Despite his busy schedule, Feifei found time to discuss my work and gave indispensable advice both about graduate school and about pursuing a career in Computer Science in addition to imparting invaluable Database wisdom for the thesis.

I only wish that I had met Chinmay Kulkarni earlier. 
I couldn't have asked for a better labmate, and I have learned a lot from him in the short time we have worked together. 
I am sure Chinmay will continue to work wonders for our group, and I look forward to his successful career. 
I would also like to thank Tian Zhang(Candy) and QingKai Lu for their legwork that founded this research. 
Graduate school would have been a lot different if not for my roommates and longtime friends, Alex Karathra and Rony Gregory. 
I thank them for all their encouragement and support. Since I don't want to avoid missing names, I would like to thank all my friends who were supportive of my decision to quit my job and come back to school.

I have made plenty of decisions in my brief academic and professional career and attending the University of Utah was the single most rewarding decision that I have made till now. 
Like any other student, I have had the pleasure of expanding my horizon of knowledge exponentially after every stage of my education. 
I would like to thank all my teachers from my early schooling days at Vijayamatha and T.H.S.S and later at Govt. Model Engineering College. 
I would like to thank my mentors at Zynga and Dreamworks Animation who taught me to push further, especially Shan Kadavil, Binu Philip, Vyas Thottathil, Manik Taneja, Blessan Abraham and Satheesh Subramanian.

This material is based upon work supported by the National Science Foundation under Grant Nos\ CNS-1566175, \ CNS-1338155.
Any opinions, findings, and conclusions or recommendations expressed in this material are those of the author(s) and do not necessarily reflect the views of the National Science Foundation.