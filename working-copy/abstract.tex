%%% -*-LaTeX-*-
%%% This is the abstract for the thesis.
%%% It is included in the top-level LaTeX file with
%%%
%%%    \preface    {abstract} {Abstract}
%%%
%%% The first argument is the basename of this file, and the
%%% second is the title for this page, which is thus not
%%% included here.
%%%
%%% The text of this file should be about 350 words or less.

%K most systems optimise for small responses, others require hardware software balance

Efficient movement of huge amounts of data over fast networks at high throughput 
is essential for a modern day In-memory storage system. Modern NIC offers interesting
opportunities and challenges in this area and we can achieve this by leveraging smart data layout
and concurrency control.

In recent years, a new class of database systems were developed in response to the growing needs of throughput
and latency demands at scale. Declining cost of DRAM and increasing access to 
high throughput, low latency network fabrics in the datacenters guided their development.

A significant fraction of these systems were developed keeping OLTP workloads in mind 
and an overwhelming number of of these systems have their underlying
implementation closer to that of a key value store to aid quick dispatch decisions. These design 
decisions have an inadvertent consequence that these systems perform disproportionately
well for smaller request response cycles. Heavy responses to workloads such as range queries
and migration of huge amounts of data for load balancing and cluster reconfiguration
poses challenges orthogonal to this design. We found that performing large data transfers in a live cluster 
of systems optimised for smaller responses has a detrimental effect to their primary intent. 
SLAs of these systems are often calculated with primary workloads and huge transfers end 
up violating them in practice. The ability to effectively transfer huge amounts of
data will add to acceptance of these systems.

We evaluated the impact of modern NICs on data layout by measuring transmit performance
while tuning the effects of DMA, RDMA and caching advancements. We proposed guidelines
leveraging a smart co-design of data layout, concurrency control and advancements to the
NICs that result in increased overall efficiency and throughput. We will also set up experiments
that outline bottlenecks in current approaches to data migration and call for fast and 
efficient migration with minimal disruption to their normal operation.