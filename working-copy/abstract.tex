%%% -*-LaTeX-*-
%%% This is the abstract for the thesis.
%%% It is included in the top-level LaTeX file with
%%%
%%%    \preface    {abstract} {Abstract}
%%%
%%% The first argument is the basename of this file, and the
%%% second is the title for this page, which is thus not
%%% included here.
%%%
%%% The text of this file should be about 350 words or less.

In recent years, a new class of storage systems were developed in response to
the growing needs of throughput and latency demands at scale. While 
database systems of the previous generation had their design considerations dominated
in large part by the slow storage mediums, the landscape has changed a lot with 
newer In-memory databases. Declining cost of DRAM and increasing access of high
throughput and low latency network fabrics in the datacenters asks a slightly 
different set of questions to the database designer. Modern NICs provide plenty 
of optimisations which can result in reduced CPU involvement in data transfer.
We evaluated the application of zero copy NIC DMA for in memory databases and 
how data layout in memory affects the speed of network transmission, which even today
accounts for a significant portion of end-to-end latency to the user. We believe
that another aspect of these newer class storage systems that are largely overlooked 
is efficient reconfiguration of the system in response to varying workloads and capacity.
While other attempts at the question has relied on scale, pre-emptive partitioning of data,
and reactive transfer to satiate hotspots, we propose a few experiments that lay the groundwork
for a more efficient class of data migration which attempts at data migration an order of 
magnitude faster than the state of the art while causing minimal disruptions to the state 
of the system. We propose some experiments that will lay the ground work for a new migration
protocol in the lens of a log structured, in-memory database systems such as RAMCloud.

