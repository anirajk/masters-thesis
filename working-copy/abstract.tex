%%% -*-LaTeX-*-
%%% This is the abstract for the thesis.
%%% It is included in the top-level LaTeX file with
%%%
%%%    \preface    {abstract} {Abstract}
%%%
%%% The first argument is the basename of this file, and the
%%% second is the title for this page, which is thus not
%%% included here.
%%%
%%% The text of this file should be about 350 words or less.

In the recent years, declining cost of DRAM and increased access to state 
of the art networking fabrics have led to the development of a new class 
of storage systems. These systems while providing phenomenal performance gains
also needs to look at prior design choices that were limited by hardware at the time.
Low overhead network transmission is a key aspect here. Modern NICs allow sophisticated
mechanisms such as kernel bypass which helps push existing latency and throughput boundaries
of distributed storage systems, but if used carelessly, but these techniques pose a threat 
if not used well. We evaluated how data layout impacts data transfer if we used conventional
copy techniques on newer hardware in contrast with the newer zero copy paradigms
Another dimension of these systems that have been largely  overlooked before is the fast 
and efficient reconfiguration of the system in response to varying workloads and capacity.
While other attempts at the question has relied on scale, pre-emptive partitioning of data,
and reactive transfer to satiate hotspots, we propose a few experiments that lay the groundwork
for a more efficient class of data migration in the lens of a log structured in memory
database system such as RAMCloud.

