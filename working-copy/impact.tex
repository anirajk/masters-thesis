%%% -*-LaTeX-*-

\chapter{System Impact}
\label{chap:impact}

We saw how the Zero Copy promises savings of resources and enhanced performance 
in the previous chapters. To put things in the context of resource utilisation and
 to quantify performance gains, we profiled our benchmark code for memory bandwidth 
 and some other interesting metrics. One interesting observation from our initial 
 attempts was that our profiling code shouldn't interfere with the timing of our highly performant 
 benchmark which churns out millions of operations per second and microsecond latencies. 
 We searchaed for light weight profiling methods which adds minimal overheads.
 We ended up using Intel\textregistered's Performance Counter Monitoring module~\cite{intelpcm}. 
 An interesting thing about measuring performance 
 of a modern CPU is that almost all of the metrics of significance exist outside of the cores.
 The newer generation CPUs since the Sandy Bridge\textregistered microarchitecture provide an uncore performance 
 monitoring unit and we can use tools such as perf~\cite{perftool} to instrument these performance 
 counters. 

\section{Memory Bandwidth}
In a modern In-Memory Database server, DRAM utilisation is the most precious resource~\cite{ramcloudfast}. 
 Available memory bandwidth is also at a premium since there is a direct correlation between Memory Bandwidth
 utilisation and DRAM latencies and starvation. Since uncore events will give the most accurate measurements,
 we explored all uncore events that induce pressure on the memory controller. We found that 
\texttt{LLC\_MISSES X 64} (cache line size) is an interesting metric, but there was a problem that it wouldn't account for prefetch 
 misses. We went for the more accurate \texttt{MEMORY\_BW\_READS} and \texttt{MEMORY\_BW\_WRITES} metrics which are derived from 
 \texttt{CAS\_COUNT.RD} and \texttt{CAS\_COUNT.WR} metrics which are the number of cache lines read and written in the sampling 
 interval. We used an interactive reporting tool written on top of perf with named mappings for these events
 to measure this metric. We sampled total memory bandwidth consumed by the system while our benchmark 
 was running with fixed record sizes, number of records and mode of copying (Zero Copy and Copy Out).

\begin{figure}[t]
\includegraphics[width=\textwidth]{fig-membw.pdf}
\caption{Measured Memory Bandwidth Impact on the system during transmission}
\label{fig:membw}
\end{figure}


Figure ~\ref{fig:membw} shows the memory bandwidth consumed by the system while transmitting data at various modes 
of copying varying record sizes and transmission sizes. We can see that using zero copy on larger transmissions involving 
larger record sizes is clearly a win. The benefits are highlighted along largest transmissions of 32 records of sizes 1~KB each. 
 We had measured that a userspace application could achieve a peak bandwidth of 24~GB/s in our system. This would mean that Copy out 
 might hurt the system to the tune of a third of all available bandwidth. This is even more interesting considering that the modern day 
 In-memory stores are not tuned for large responses. When the DRAM latency and accesses go up, the performance of the whole system degrades.

For smaller data sizes, it's even more interesting. Zero Copy doesn't show a huge contrast in memory bandwidth utilisation. There are times 
when Copy out outperforms Zero Copy even in terms of memory bandwidth utilisations. Its interesting to note that these data points overlap the anomalous transmit performance 
degradation we saw while observing transmission throughput in Figure~\ref{fig:zero-copy-tput}. Interestingly, after this region in the graph 
where you transmit 16 or more 128~B records where Zero Copy offers better transmission throughput, the memory bandwidth consumption by Zero Copy 
is reduced and Copy-out starts to appear more expensive in terms of memory bandwidth. 

\subsection{Memory Bandwidth and Transmission throughput}
\begin{figure}[t]
\includegraphics[width=\textwidth]{fig-membw-ratio.pdf}
\caption{Ratio of Memory Bandwidth to Transmission Throughput}
\label{fig:membw-ratio}
\end{figure}

Figure~\ref{fig:membw-ratio} shows the ratio of Memory pressure exerted per bytes transmitted. We can clearly see the value proposition of 
Zero Copy mode of transmitting here. For Zero Copy, We can see that the system is consistent and predictable and only incurs a tiny overhead of 
copying descriptors other than DMAing the data on to the NIC. The overhead of the descriptors is evident from the fact that 128~B records show 
a bigger multiplier on transmission throughput than 1024~B records. For 128~B records, the ratio appears tumultous across varying number of records, 
 but still shows Zero Copy as the clear winner. If we were to compare larger 1~KB records across the two modes of copying, we can see that there is a 
 2X memory bandwidth cost on copying out 1024~B records.

\section{DDIO traffic}
\begin{figure}[t]
\includegraphics[width=\textwidth]{fig-ddiobw.pdf}
\caption{LLC misses from DDIO vs Throughput}
\label{fig:ddiobw}
\end{figure}

Newer generation of Intel Processor's have a portion of their Last Level Cache dedicated for I/O. This was initially called Direct Cache Access\textregistered 
and more recently named DDIO\textregistered. This 

\section{Anomaly in Transmission Throughput}
\begin{figure}[t]
\includegraphics[width=\textwidth]{fig-pciebw.pdf}
\caption{LLC misses due to PCIe vs Throughput}
\label{fig:membw}
\end{figure}

