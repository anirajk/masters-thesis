%%% -*-LaTeX-*-

\chapter{Introduction}
\label{chap:intro}
For decades, the performance characteristics of storage devices have dominated
 thinking in database design and implementation. From data layout to
concurrency control, to recovery, to threading model, disks touch every aspect
of database design. In-memory databases effectively eliminate disk Input/Output (I/O) as a
concern, and these systems now execute millions of operations per second,
sometimes with near-microsecond latencies.  It is tempting to believe 
database I/O is solved; however, another device now dictates
performance: the Network Interface Card (NIC). As the primary I/O device in present-day databases,
NIC should be a first class citizen in all phases of database design.
However, modern network cards have grown incredibly complex, partly in response
to demands for high throughput and low latency.

We set out to improve the efficiency of RAMCloud's~\cite{ramcloud} data migration mechanism. 
RAMCloud is a storage system that keeps all data in DRAM at all times and provides low latency access by
leveraging advancements like kernel bypass which are present in a modern NIC.
To thoroughly understand the complexities involved in the NIC in doing huge
transfers of data, we profiled a modern NIC with a focus on massive data transfers.
 We made several discoveries that informed our design for RAMCloud,
 but they also generalize to other operations involving large transfers of data in a distributed in-memory database. 

This thesis is structured into six key pieces. 
Chapter~\ref{chap:intro} gives an introduction to the problem space and the motivation for this work. 
It also outlines an overview of all the experiments and key contributions. Chapter~\ref{chap:modernnics} introduces 
our notion of \enquote{\textbf{Zero Copy}} and \enquote{\textbf{Copy Out}} and explains 
the lessons we learned when we developed a microbenchmark to profile a modern NIC with kernel bypass and RDMA doing bulk transfers 
of data. The chapter also provides a detailed study of the impact on the available system resources while we achieve 
near the line rate of network transmission performance. Recommendations for a database designer are provided, and takeaways 
from our experiments are explicitly listed. Chapter~\ref{chap:applications} enumerates a few scenarios where we can 
leverage our observations to improve server efficiency and network transmission using a client assisted design.
 We quantify the benefits of such an approach and discuss future directions on how this design will apply to column-oriented 
 databases. Chapter~\ref{chap:migration} discusses how the nuanced performance characteristics for the NIC can influence 
design for rapid reconfiguration in RAMCloud~\cite{ramcloud}. It discusses a breakdown of RAMCloud's existing migration mechanism to understand its bottlenecks
and proposes the basics of a new design for migration that we believe will advance state of the art
in the reconfiguration of distributed in-memory databases. Chapter~\ref{chap:relatedwork} discusses prior work that laid the 
foundation for this research and the state-of-the-art in in-memory databases and data transfer in high-speed networks. 
Chapter~\ref{chap:conclusion} reiterates our conclusions and discusses future directions for this work.

While developing the microbenchmark to analyze a NIC, one thing was becoming increasingly clearer to us.
Modern network cards have grown incredibly complex, partly in response to
demands for high throughput and low latency. One key advancement these NICs made 
is the presence of some form of kernel bypass where NIC Direct Memory Access (DMA) engine can be 
controlled by the application and the network card can access application memory without invoking the CPU. 
This development led to the Remote Direct Memory Access (RDMA)~\cite{rdmapatent} paradigm. 
The modern NIC with its complex architecture makes understanding its characteristics difficult, 
and it makes designing software to take advantage of them even harder. 
Database designers will encounter many trade-offs when trying to use the NIC effectively. 
This thesis helps navigate the complex trade-off space with concrete suggestions backed by measurements.

Our mission was to come up with a migration protocol that advances state of the art for in-memory storage systems.
We realized the importance of the making data transfer NIC friendly and set out to measure what variables impact the 
transmission performance and system stability. We wanted to enumerate and underline the challenges in designing keeping NIC as a first class citizen.
We specifically looked at five factors that make it especially challenging to design NIC-friendly
database engines:
\begin{itemize}
  \item {\em Zero-copy} DMA engines reduce server load for transferring large data
    blocks, but large, static blocks are uncommon for in-memory databases,
    where records can be small, and update rates can be high.
  \item The performance gains from zero-copy DMA do not generalize to
      {\em finer-grained objects} which are commonplace in in-memory stores for two reasons:
      \begin{enumerate}
        \item Transmit descriptors grow linearly in the number of records;
        \item NIC hardware limits descriptor length, restricting speed for small records.
      \end{enumerate}
      Despite this, we find that zero-copy can make sense for small objects under certain conditions, such as
      small ``add-ons'' to larger transmissions.
   \item Absolute savings in {\em consumed memory bandwidth} is far greater than the CPU savings 
       offered by Zero Copy
   \item Zero-copy introduces complications for \emph{locking and object
      lifetime}, as objects must remain in memory and must be
      unchanged for the life of the DMA operation (which includes transmission
      time).
  \item {\em Direct Cache Access}, now rephrased as Intel\textregistered Data Direct I/O~\cite{ddio},
      provide direct access to last level cache from an I/O device. This effect
      impacts the consumed memory bandwidth significantly.
\end{itemize}


These factors make advanced NIC DMA difficult to use effectively. We set out to explore data 
transmission purely from the perspective of a modern NIC. We profiled a cluster of machines with
Infiniband ConnectX\textregistered-3, a modern kernel-bypass capable NIC
with RDMA capabilities, with specific attention to large transfers and query responses.
We published our findings and made some concrete suggestions around how NICs might
influence data layout and concurrency control in a modern in-memory 
database~\cite{imdmpaper}. 

Our primary results gave us ideas about how the Zero Copy paradigm will perform against
the traditional approach of transmitting via larger buffers. We also made some optimizations to our 
benchmarks to better leverage the effects of DDIO~\cite{ddio} which are explained in Section ~\ref{sec:code-optimisation}. 
We also measured memory bandwidth and a few other metrics with the help of Intel\textregistered's Performance Monitoring Unit (PMU).

Another operational burden while using these newer in-memory storage systems is to find a way to 
provide effective reconfiguration. Our results from the benchmark highlighted the importance of 
hardware considerations in data migration. State of the art systems like RAMCloud have around 1000$\times$
lower latency than traditional storage systems, and these tight SLAs make it harder to do migrations
without interfering with regular requests. The need for rapid reconfiguration with minimal system 
impact is one of the key hurdles to overcome before the mainstream acceptance of large scale 
in-memory storage systems.

Our findings on modern NICs inform a new design for cluster reconfiguration
in the RAMCloud storage system. Low-cost load balancing improves RAMCloud’s
efficiency since indexes~\cite{slik} and transactions~\cite{ramcloudtx} benefit from collocation between servers.
This is tricky to implement and RAMCloud is designed with no locality within a 
server to provide high memory utilization~\cite{ramcloudfast} via efficient log compaction.
The more interesting engineering detail is RAMCloud’s exceedingly tight SLAs (median
access time of 5\textmu~s and 99.9\% of reads within 10\textmu~s). It would be great if we can saturate the
available NIC bandwidth which is 8-12GB/s~\cite{cx3,cx4} while state of the art 
transfer mechanisms offers many orders of magnitudes lower than that at ~10~MB/s 
throughput while operating at latencies that are 1000-10000$\times$ higher than that of RAMCloud~\cite{ramcloud}.
We wanted our protocol to strike a balance between fast operation and minimal disruption without drastically affecting
the latency and throughput available for the system. The following key principles guide our design for fast and 
efficient migration for RAMCloud:
\begin{itemize}
\item{\textbf{High Throughput}}: State of the art systems do migrations at 5~-~10 MB/s. Our proposed protocol
should result in 100-1000x faster data migrations. We aspire to have rapid reconfiguration that will 
help with enable frequent reconfiguration on live clusters.
\item{\textbf{Low Impact on latency}}: RAMCloud has 99.9 percentile read latency at $<$10\textmu~s and write latency
at $<$100\textmu~s. Our proposed protocol will be of minimal impact to the normal case operation, still keeping
up with SLAs that are 100x tighter than other systems~\cite{squall}.
\item{\textbf{Judicious use of system resources}}: In an in-memory store, the significance of resources such as DRAM capacity and
memory bandwidth are paramount. Our protocol should make the most efficient use of these resources.
\item{\textbf{Late-partitioning}}: RAMCloud promises uniform access latency regardless of the level of data locality.
RAMCloud also doesn't need to pre-partition data for normal operations. Most other systems diverge from this 
approach to make data movement faster. An ideal solution should provide uniform access latency and high DRAM utilisation 
without needing early partitioning decisions.
\end{itemize}

We evaluate the current state of the art approaches to migration in RAMCloud storage system.
We lay out some experiments that highlight issues with current approaches and motivates our new 
fast and efficient migration protocol. To motivate and demonstrate our approach, we explore
the following experiments:
\begin{itemize}
\item Measuring impacts of {\em locality of data} in RAMCloud; we set up an experiment to show how 
the locality of data affects the collective load and throughput on the system.
\item {\em Factor analysis of migration bottlenecks}; RAMCloud's current migration protocol, while orders 
of magnitude faster than existing state of the art protocols, leaves more to be desired of in terms of achievable perforamnce.
We analyse the various bottlenecks in the current protocol and the performance benefit we gain by enumerating savings on avoiding 
each one.
\end{itemize} 

\section{Contributions}
\label{sec:contributions}
\textbf{\enquote{Scale-out in-memory stores are optimized for small requests
under tight SLAs, and bulk data movement, for rebalancing and range queries, interfere;
The thesis argues carefully leveraging data layout and advancements in modern NICs
will yield to gains in performance and efficiency for large transfers in these systems
without disrupting their primary obligations.}}
 
We have shown that careful co-design of data layout, concurrency control
and networking can lead up to 95\% reduction of CPU overhead. 25\% absolute CPU savings and a 
50\% reduction in consumed bandwidth all the while getting a throughput
boost of 1.6$\times$ compared to blindly using the kernel bypass available
in the NICs. We also profiled the bottlenecks in the way data migration is set up
in RAMCloud and in conjunction with the benchmarks. Though there were efforts to 
dismantle cost of a database in the network layer before, the server-side impact
on mixed workloads including range scans and migrations have not been well understood
until now. In the light of our discoveries, we make concrete suggestions on how
 to design an fast and efficient migration protocol which causes minimal disruption.

The following are the main contributions from this work which supports and provides
evidence for the thesis statement in the beginning of this section.

\begin{itemize}

  \item{\textbf{Evaluation of Influence of Modern NICs on data layout}}: To our knowledge,
   this is the first published research exploring the scatter gather DMA engine of the NIC 
   in contrast with the traditional approach of transmission the context of bulk transfers in in memory databases. 
   We profiled the transmit performance of a modern NIC in depth to show the influence of NICs on data layout. 
   We have found a few arguments against the blind use of the new kernel bypass approach considering
   the overhead of concurrency management and difficulty of implementation.
   \begin{itemize}

    \item Developed a microbenchmark to assess performance of Infiniband verbs

    \item Measured transmission throughput and CPU overhead in the face of varying
    data layouts

    \item Compared the performance of Zero-Copy and Copy Out mechanisms of 
    transmitting data from an RDMA capable modern NIC.
   \end{itemize} 

    \item{\textbf{The design of client assisted responses for better transmit performance}}: Our measurements inform a new design for migration in RAMCloud, but it also has impacts on other database operations.
     While evaluating the results, we saw that there are certain structures and circumstances that can uniquely take advantage of the hardware limited scatter gather entries of a modern NIC.
     Chapter~\ref{chap:applications} explores ways in which a modern B-tree index can exploit NIC hardware.
     Through the benchmark, we show 60\% improvement in throughput using such a design with a 30$\times$ reduction in CPU overhead.


    \item{\textbf{Evaluation of server impact during transmission}}: We measure performance counters outside of CPU cores that reveal the impact on memory bandwidth,
     DDIO traffic and Peripheral Component Interconnect Express (PCIe) traffic in a system that's churning out heavy responses. This is the first study to our knowledge that does an 
     in-depth analysis of server impact using uncore events for large data transmissions near line rate in a kernel bypass capable modern NIC.
     \begin{itemize}
     \item Profiled the number of cache lines read and written (including the pre-fetcher effect) to measure consumed memory bandwidth.
     \item Profiled the number of DRAM accesses from Last Level Cache(LLC) that results from DDIO.
     \item Profiled the number of DRAM accesses from Last Level Cache(LLC) that results from PCIe errors.
     \end{itemize}


  \item{\textbf{Experiments to motivate fast and efficient reconfiguration}}: This is the first assessment of data migration on SLAs for a high performance,
    in-memory store that leverages high throughput, kernel-bypass networking fabrics. 
   We laid out experiments that show how even the state of the art systems for cluster reconfiguration becomes suboptimal with its locality constraints and overall slowness. 
   RAMCloud assumes uniform access cost to all nodes. We quantify the gains from explicitly colocating records.
   Our experiments call for fast and efficient migration protocol especially suited for an In-memory database system with a decentralized log structured data model such as RAMCloud.

  \item{\textbf{Preliminary design for a migration protocol}}: After careful analysis of the current migration protocol
   we discovered the critical bottlenecks in the current design that makes it infeasible for fast and efficient migration.
   We propose the basics of a migration protocol that will effectively eliminate these bottlenecks and pave the way
   forward for fast and efficient transfer of huge chunks of data.

\end{itemize}

