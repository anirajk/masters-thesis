%%% -*-LaTeX-*-

\chapter{NIC Aware Reconfiguration}
We have seen that the current generation of NICs are capable of transferring Gigabytes of 
data per second in latencies of the order of a few microseconds and we have reasons to believe 
that the next generation of networking hardware will have the ability to process 
hundreds of millions of messages per second at bandwidth of the order of tens of Gigabytes per 
second and sub microsecond latencies\cite{cx6}.
We saw from the previous chapters how the nuanced performance characteristics of 
a modern NIC will influence overall transmit performance and server resource utilisation 
for large data transfers such as responses to range scans. Yet another pressing 
concern for a modern In-memory database operating at scale is quick reconfiguration. 
Since DRAM is the most expensive resource in a modern cluster, efficient rebalancing 
involves fast reconfiguration that is frugal on resources and keeps the impact on their 
SLAs minimal. Informed with our lessons about the NIC, it is reasonable to assume that protocols for reconfiguration 
today can benefit from these optimisations and armed with the changes for proposed protocol, data migration might come 
closer to what the hardware is capable of doing today.

In this Chapter, we discuss the state of the art migration protocols and their 
performance, does an analysis of the existing migration protocol in RAMCloud, 
and armed with the lessons learned by profiling a modern NIC, we make concrete 
suggestions on how a new protocol might be well poised to take advantage of the 
new hardware.


\section{State of the art}
Fast reconfiguration of In-memory data is something that has been overlooked largerly
by the research community. There has been novel techniques for live migration of 
key value stores such as Albatross~\cite{albatross} which provides unavailability windows 
as short as 300ms, Zephyr~\cite{zephyr} which does on-demand pulls and asynchronous data 
pushes to minimize failed transactions and unavailability for migration while operating 
at transactional latencies of 50ms. Squall~\cite{squall} went further with fine grained 
partitioning with optimisations such as pre-fetching. Squall also offers minimal overhead 
to their transactional latencies which are of the order of tens of milliseconds and migrates 
data overall at the rate of around 5MB/s. While these efforts offer reconfiguration speeds 
better than other systems and minimal addition to their millisecond latencies, that is far 
from what a modern NIC is capable of delivering. From our benchmark, we knew that the NIC 
can deliver transmission throughputs of around 5GB/s while operating at microsecond latencies. 
There are important lessons to take from these recent research efforts on migration, but without 
devoting attention to the NIC and it's complex performance characteristics, today's cluster of
storage servers cannot offer migration performance order of magnitudes better than the state of the art.

\section{Migration in RAMCloud}
We saw from previous section what the latest reconfiguration protocols are capable of doing and how they can only 
offer 1000$\times$ less throughput than the latestoperating under millisecond latencies. There is a different class of systems~\cite{ramcloud,farm,rdmabillion,herd} that 
were developed recently which optimises for end-to-end latency and has the ability to perform millions of operations 
per second per server. RAMCloud~\cite{ramcloud} is a prime example of such a system which promises median read latencies 
of 5$\mu$s and responds to workloads upto a million operations per second. These characteristics result from the fact that 
these systems were designed from ground up keeping low latencies in mind and any significant disruption to their operating 
latencies might deem them less useful. RAMCloud has a migration protocol which offers around 100MB/s data transfer speeds albeit 
synchronous and unpipelined. While this is better by leaps and bounds than other systems, we believe that carefully leveraging the NIC, 
we can attain better throughput without compromising on latency.

\subsection{Current Protocol}
\begin{figure}[t]
\includegraphics{fig-current-migration.pdf}
\caption{Diagram showing steps involved in the current migration protocol of RAMCloud}
\label{fig:migration-current}
\end{figure}

Figure ~\ref{fig:migration-current} shows the steps involved in the existing protocol. The key drawback to the current approach is that 
most of the steps involved are synchronous and blocking. We analysed 

\subsection{Locality in RAMCloud}


\section{Design Guidelines for a new protocol}
\begin{figure}[H]
\includegraphics{fig-bottlenecks.pdf}
\caption{Bottleneck plot}
\label{fig:bottlenecks}
\end{figure}

\begin{figure}[t]
\includegraphics{fig-colocation.pdf}
\caption{Colocation plot}
\label{fig:colocation}
\end{figure}
