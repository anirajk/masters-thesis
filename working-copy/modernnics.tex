%%% -*-LaTeX-*-
\chapter{Modern NICs}
\label{chap:modernnics}
Network Interface Cards have grown complex and present opportunities and challenges as part of a modern distributed system.
 It is imperative that today’s database system designer considers the network in design decisions.
 Previous research for optimising databases were focussed more on processing time spent on a database query~\cite{dbmsproctime},
  but today’s truth is that most of the end-to-end latency while processing a query is spent in various parts of the networking stack~\cite{ramcloudosr}.
Since In-Memory Databases became popular and datasets started to fit in memory, the spinning disks 
which were the primary I/O bottlenecks before aren't anymore. The bulk of I/O overhead in a distributed storage system today is in network transmission.
Adding this to the fact that the modern NIC allows us to offload CPU from the data transfer path,
the saved memory bandwidth and CPU cycles could be utilised for doing useful work.

Hardware overheads have thinned a lot over time and system researchers had to dig deeper to target number to minimize CPU interrupts and come up with forms of kernel by-pass networking to extract more performance.
 Network cards have had TCP offloading baked in as TCP Offload Engine for some time now, but it was getting more and more ineffective because of performance issues and complexities involve~\cite{tcpoffload}. 
 RDMA~\cite{rdmapatent,rdmacase,rdma} was proving to be a much better alternative where TCP offloading was just one among various benefits involved.
  Development of advanced host-device interconnects such as PCI Express~\cite{pcie}  made it possible to develop highly performant network devices offering throughputs and latencies a couple of orders of magnitudes better than what was possible before.

 The key consideration we explore in this chapter is how to make the best use of the NIC’s ability to receive and transmit data directly at the application level, especially for the
cases when the system must move large chunks of data. We explore the various trade-offs and design dimensions and provide our evaluation of those choices with the help of a microbenchmark.

\section{Zero Copy and Copy Out}
Kernel bypass is a ubiquitous feature in the modern NIC and key to accessing remote memory directly 
from across the network. Literature defines kernel-bypass as any data movement that doesn't 
involve copying between user space and kernel space. Newer network controllers make use of vectored I/O, otherwise known as scatter/gather I/O
where a single procedure reads data from multiple buffers and writes it to single buffer. The Mellanox Infiniband ConnectX\textregistered-3
NIC that we profiled provides a scatter/gather list which powers it's own DMA engine.

Before kernel bypass became commonplace, the steps for copying data over the network using socket programming was as follows. 
The application calls a send function in a socket library and the data for sending is assembled 
in a transmit buffer, this transmit buffer is copied on to the on-NIC buffers and the data is 
send over the wire. We could employ the same technique in the kernel bypass capable NIC as well.
Instead of employing a multi entry scatter/gather list, the data to be transmitted, which could be discontiguous in memory, could be assembled on 
to a temporary buffer and this buffer could be treated as a single entry scatter/gather list
and then transmitted over the network. We will be calling this technique \enquote{\textbf{Copy Out}} for the remainder of the thesis. 

On the other hand, if we were to ditch the idea of conventional copy and instead exploit the scatter/gather list to 
transmit data directly from where they live in memory, we could reduce the number of copied involved. This style of transmitting data is 
what we'll refer as \enquote{\textbf{Zero Copy}} from now on.



Zero copy DMA facilitates a scatter gather list of buffer descriptors
which could be mapped to non contigous locations in memory on demand. These provide
the added benefit that network headers don't need to exist along with data and helps bring more flexibility 
in deciding the transport layer. One subtle thing to note here is that Zero Copy only implies the absence 
of a memcopy from the records to a buffer. The data still needs to be copied to the on-NIC buffers before 
it is sent out via the network cable. This is different from what happens when we call socket send on a traditional stack where the data for transmission is pre assembled in a huge buffer and then copied across the network.
We evaluate how Zero Copy performs in contrast with the more traditional Copy Out from now on.

\section{Memory Bandwidth}
Interestingly, the additional copy involved in the traditional approach hurts memory bandwidth of the system more than contributing 
to additional CPU load purely from the perspective of network transmission. Our evaluation show us that while the traditional
copy adds upto 25\% increase in absolute CPU utilisation, making use of the high throughput available in NICs by transmitting
near line rate takes up one fourth of the total available memory bandwidth in a modern server. We should read this in the context that 
network transmission is not the primary responsibility of a storage server in a distributed system and costs of communication should be treated as overhead.
 Most of the memory bandwidth is wasted in just aggregating the data before transmission could be used as part of actual computation or other useful rearrangement of data in 
the query response. Chapter \ref{chap:impact} will fully discuss the impact on memory bandwidth and other parameters while the system 
is transmitting large amounts of data.


\section{NIC structures in detail}
\begin{figure}[t]
\includegraphics[width=\textwidth]{fig-mem-regions.pdf}
\caption{Key structures involved in network transmission.}
\label{fig:mem-regions}
\end{figure}

Figure~\ref{fig:mem-regions} details how an application interacts with a Mellanox
ConnectX-3\textregistered , a modern 56~Gbps NIC that uses kernel bypass. 
With zero-copy, transmit descriptors list several chunks of data for
the NIC to DMA. With copy-out, all data to be transmitted is first explicitly
copied into a transmit buffer by the host CPU; then, a transmit descriptor is
posted that references just the transmit buffer rather than the original
source data. Both zero-copy and the traditional copy-out approaches to transmission are shown.
In both cases the same three key data structures are involved. The first important structure is the
data to be transmitted, which lives in heap memory.  For zero-copy, the memory
where the records live must first be registered with the NIC. Registration
informs the NIC of the virtual-to-physical mapping of the heap pages. This is
required because the NIC must perform virtual-to-physical address translation
since the OS is not involved during transmission and the application has no
access to its own page tables.  Registration is done at startup and is often
done with physical memory backed by 1~GB hugepages to minimize on-NIC address
translation costs.

The second key structure is the descriptor that a thread must construct to
issue a transmission. With Mellanox NICs, a thread creates a work request and a
gather list on its stack. The work request indicates that the requested
operation is a transmission, and the gather list is a contiguous list of
base-bound pairs that indicate what data should be transmitted by the NIC (and
hence DMAed). For zero-copy, the gather list is as long as the number of chunks
that the host would like to transmit, up to a small limit. The NICs we use support
posting up to 32~chunks per transmit operation. Later, we find that this small
limit bottlenecks NIC transmit performance when chunks are small and numerous.

The final important structure is the control interface between the NIC and the
host CPU.  When the NIC is initially set up by the application, a region of the
NIC's memory is mapped into the application's virtual address space. The NIC
polls this region, and the host writes new descriptors to it from the thread's
stack to issue operations. The region is mapped as write-combining; filling a
cacheline in the region generates a cacheline-sized PCIe message to the NIC.
The NIC receives it, and it issues DMA operations to begin collecting the data
listed in the descriptor. The PCIe messages are posted writes, which means they
are asynchronous from the CPU's perspective. Even though PCIe latencies are much
higher than DRAM access, the CPU doesn't stall when posting descriptors, so the
exchange is very low overhead.

\subsection{Zero Copy vs Copy Out}
The key difference between zero-copy and copy-out is shown with the wide, red
arrows in Figure~\ref{fig:mem-regions}. Copy-out works much like conventional
kernel-based networking stacks: chunks of data are first copied into a single
transmit buffer in host memory. Then, a simple, single-entry descriptor is
posted to the NIC that DMAs the transmit buffer to an on-device buffer for transmission
As a result, copy-out requires an extra and explicit copy of the data, which is made
by the host CPU.  Making the copy uses host CPU cycles, consumes memory
bandwidth, and is pure overhead. Surprisingly, though, copy-out has
advantages including better performance when
records are small and scattered.  In those cases, complex gather descriptors
bottleneck the NIC, and using the host CPU to pre-assemble the responses can
improve performance.



\section{DDIO}
One conflating factor in understanding the benefits of Zero Copy is Data Direct I/O (DDIO)~\cite{ddio}, which is a performance-oriented enhancement 
to the DMA mechanism have been introduced in Intel\textregistered Xeon E5 processors with DDIO feature,
allowing the DMA ``windows" to reside within CPU caches instead of system RAM. As a result,
CPU caches are used as the primary source and destination for I/O, 
allowing network interface controllers (NICs) to talk directly to the last level caches of local CPUs
and avoid costly fetching of the I/O data from system RAM. As a result, if we have enough locality of data to exploit,
DDIO reduces the overall I/O processing latency, allows processing of the I/O 
to be performed entirely in-cache and prevents the available memory bandwidth from becoming a performance bottleneck.
We fully investigate the effects of DDIO in Chapter~\ref{chap:impact} in order to conclude the impact of 
traditional copy mechanisms.


\section{Inlining}
Mellanox NICs allow some data to be {\em inlined} inside the control message
sent to the NIC over PCIe. Our NICs allow up to 912~B to be included inside the
descriptor that is posted to the NIC control ring buffer.  Inlining can improve
messaging latency by eliminating the delay for the NIC to DMA the message data
from host DRAM, which can only happen after the NIC receives the descriptor.
Inlining benefits small request/response exchanges, but it does not help for
larger transmissions. This is because even though there is an extra delay
before the NIC receives the actual record data, that delay can be overlapped
with the DMA and transmission of other responses. Other researchers have shown
that sending data to the NIC via MMIO also wastes PCIe bandwidth~\cite{rdma}.
All of our experiments have inlining disabled. Enabling inlining gives almost identical throughput and overhead, except it only
works for transmissions of ~912 B or less.

\section{Fine grained concurrency}
After we published our findings~\cite{imdmpaper}, we profiled our code and it was becoming clear to us that most of the time spent 
in running our benchmark was spent on busy waiting. We employed a more fine grained 
locking with the use of thread local storage for our buffers and time spent on 
contention and overall throughput improved considerably. The measurements in this thesis only include the
data after the optimisation was made.

\section{Evaluation}
% TODO
% Should we put the tx buffers in huge pages?
% Knowing measured peak mem bw of our machines would be good, but it looks like
%   not all channels are populated?


\begin{table}[t]
\def\arraystretch{1.25}%  1 is the default, change whatever you need
\caption{Experimental cluster configuration.}
\begin{tabular}{@{}l@{\hskip 12pt}l@{}}
\toprule
\textbf{CPU} & Intel Xeon E5-2450 (2.1~GHz, 2.9~GHz Turbo) \\
    & 8 cores, 2 hardware threads each \\
\textbf{RAM} & 16~GB DDR3 at 1600~MHz \\
\textbf{Network} & Mellanox MX354A ConnectX-3 Infiniband HCA (56 Gbps Full Duplex) \\
        & Connected via PCIExpress 3.0 x8 (63~Gbps Full Duplex) \\
%        & 7 Mellanox SX6036G FDR Switches \\
\textbf{Software} & CentOS 6.6, Linux 2.6.32, gcc 4.9.2, libibverbs 1.1.8, mlx4 1.0.6 \\
\bottomrule
\end{tabular}
\vspace{0.25eX}
\label{tbl:config}
\end{table}


\subsection{Experiment Setup}

We explored hows the different designs trade-off database server efficiency and
performance, by building a simple model of an in-memory database system that
concentrates on data transfer rather than full query processing. In all experiments,
one node acts as a server and transmits results to 15 client nodes.
Our experiments were run on the Apt~\cite{Ricci+:OSR15} cluster of the
CloudLab~\cite{Cloudlab:URL} testbed: this testbed provides exclusive bare-metal
access to a large number of machines with RDMA-capable Infiniband NICs.
The proposed work that is pending will give more conclusive evidence for the 
Table~\ref{tbl:config} shows the Experiment Setup for the cluster where we profiled
the Mellanox Infiniband ConnectX-3 \textregistered NIC and impact of data layout and 
use of no update in place structures on transmission throughput. The cluster has 7
Mellanox SX6036G FDR switches arranged in two layers. The switching fabric is
oversubscribed and provides about 16~Gbps of bisection bandwidth per node
when congested. All of the experiments are
publicly available online\footnote{\url{https://github.com/utah-scs/ibv-bench}}.




\begin{figure}[t]
\includegraphics{100B_transrate.pdf}
\caption{The green dot shows the performance for RDMA reads which is also unfair
since RDMA reads can only transmit a single record at a time. The yellow dot
shows RDMA write performance. The modes shown was an enum of the format 
\cpp{enum Mode { MODE_SEND, MODE_WRITE, MODE_READ}}
\cpp{<infiniband/verbs.h> supports the various modes}
}
\label{fig:100B_transrate}
\end{figure}

\begin{figure}[t]
\includegraphics[width=\textwidth]{1000B_transrate.pdf}
\caption{Transmission rate for 100 B records over different RDMA verbs and 
various lengths of the scatter gather employed.
The green dot shows the performance for RDMA reads which is also unfair
since RDMA reads can only transmit a single record at a time. The yellow dot
shows RDMA write performance. The modes shown was an enum of the format 
\cpp{enum Mode \{ MODE_SEND, MODE_WRITE, MODE_READ\}}
\cpp{<infiniband/verbs.h> supports the various modes}
}
\label{fig:1000B_transrate}
\end{figure}


\subsection{RDMA modes}
We started out by getting measurements of transmission speeds using RDMA Reads.
We implemented rdma operations using infinibands ib verbs library. We implemented 
RDMA reads and write as op codes on the \cpp{ibv_send_wr} request which 
transmit data as \cpp{ibv_post_send} and also tried to measure the transmit
performance while using zero copy using \cpp{IBV_WR_SEND} as the operation.
We found that send operations can transmit multiple transmit buffer
It became evident quickly that one sided RDMA reads are not well positioned to 
take advantage of the zero copy paradigm since it only supports a remote read 
from a given location from the source. The rest of the experiments concerning
the impacts of NIC's impact on data layout were all done with send as the primary
operation. 


Figure~\ref{fig:100B_transrate} shows a comparison of transmission
peformance of various verbs transmitting 100 Byte records varying the number of records
that could be transmitted at a time for various operations.The green dot shows
the performance for RDMA reads which is also unfair
since RDMA reads can only transmit a single record at a time. The yellow dot
shows RDMA write performance. The modes shown was an enum of the format 
\cpp{enum Mode \{ MODE_SEND, MODE_WRITE, MODE_READ\}} and infiniband 
(\cpp{<infiniband/verbs.h>}) verbs library provides the  The record sizes were 
also an interesting measure to arrive at. It was obvious even from an earlier stage
that higher data sizes such as shown in Figure~\ref{fig:1000B_transrate} almost 
always results in better throughput measurements. In the world of In-memory databases,
there is an argument to be made that record sizes will be getting smaller since 
database designers can aggresively normalise and this has proven increasingly to
the liking of companies that manage large volumes of data in memory~\cite{fb-memcache,fb-workload}.
\setlength{\fboxsep}{10pt}
\setlength{\fboxrule}{1pt}
\fbox{Of the various modes of transmissions, send is the most promising}

\subsection{Paging}
All our experiments transmit from a large region of memory backed by 4~KB pages
that contains all of the records. The region is also
registered with the NIC, which has to do virtual-to-physical address
translation to DMA records for transmission.
In some cases, using 1~GB hugepages reduces translation look aside buffer
(TLB) misses. We have realised that the NIC can benefit from
hugepages as well, since large page tables can result in additional
DMA operations to host memory during address translation~\cite{farm,rdma}. For
our experiments, the reach of the NIC's virtual-to-physical mapping is
sufficient, and hugepages have no impact on the results.


To explore how different designs trade-off database server efficiency and
performance, we built a simple model of an in-memory database system that
concentrates on data transfer rather than full query processing.  In all experiments, one node acts
as a server and transmits results to 15~client nodes.

Experiments transmit from a large region of memory backed by 4~KB pages that contains all of
the records.  The region is also
registered with the NIC, which has to do virtual-to-physical address
translation to DMA records for transmission.
In some cases, using 1~GB hugepages reduces translation look aside buffer
(TLB) misses.  The NIC can benefit from
hugepages as well, since large page tables can result in additional
DMA operations to host memory during address translation~\cite{farm,rdma}. For
our experiments, the reach of the NIC's virtual-to-physical mapping is
sufficient, and hugepages have little to no impact on the results. Even then we 
enabled 1~GB hugepages and registered a 4~GB chunk of the memory with the NIC 
for all our experiments.

\subsection{Zero-copy Performance}
\label{sec:zero-copy-tput}

% stutsman: some old dead text, probably totally worthless.
% At some point make sure the rest of the text contains all of these ideas
% already.
%
% Smaller transmissions require more NIC interaction and PCI Express (PCIe)
% writes to transmit query results, but they can deal with discontinuous
% in-memory data layouts. In the case of zero-copy transmission, smaller
% transmissions may also reduce the amount of time that the database must keep
% records stable for the NIC.  Larger transmissions can also accommodate
% discontinuity through two different methods: copy-out or zero-copy, but
% discontinuity results in increased transmit descriptor size and complexity.

The first key question is understanding how database record layout affects the
performance of the transmission of query results.  The transmission of large
result sets presents a number of complex choices that affect
database layout and design as well as NIC parameters.  Range query
results can be transmitted in small batches or large batches and either via
copy-out or zero-copy.

To understand these trade-offs, we measure the aggregate transmission
throughput of a server to its 15~clients under several
configurations.  In each experiment, the record size, $s$, is set as either 1024~B or
128~B. Given a set of records that must be transmitted, they are then grouped
for transmission. For zero-copy, an $n$ entry DMA gather descriptor is created
to transmit those records where $ns$ bytes are transmitted per NIC transmit
operation. For copy-out, each of the $n$ records is copied into a single
transmit buffer that is associated with a transmit descriptor that only points
to the single transmit buffer. Each transmission still sends exactly $ns$
bytes, but copy-out first requires $ns$ bytes to be copied into the transmit buffer.
We vary $n$ from 1 to as much as the NIC can support DMAing in a single transmission.
Intuitively, larger groups of records (larger sends) result in less host-to-NIC
interaction, which reduces host load and can increase throughput; the benefits
depend on the specific configuration and are explored below.

\begin{figure}[t]
\includegraphics{fig-zero-copy-tput.pdf}
\caption{Transmission throughput when using conventional copy-out into transmit
buffers and when the NIC directly copies records via DMA (zero-copy). The
line for 128-byte zero-copy stops at 4K, as this is the maximum size of a send
with 32 records of this size. }
\label{fig:zero-copy-tput}
\end{figure}

Figure~\ref{fig:zero-copy-tput} shows how each of these configurations impact
transmission throughput. For larger 1024~B records, using the NIC's DMA engine
for zero-copy shows clear benefits even when you are transmitting less number of 
records. This is in addition to the CPU and memory bandwidth savings, which 
we explore in \S\ref{sec:overhead}). The database server is able
to saturate the network with zero-copy so long as it can post 2 or more
records per transmit operation to the NIC (that is, if it sends 2~KB or larger
messages at a time). For the copy out approach, you get roughly the same transmission 
throughput as Zero Copy when your transmission sizes get bigger at around 16~KB. In 
the impact study in Chapter ~\ref{chap:impact} we explain why this is not a good value 
proposition as well as the dip in transmission throughput when we go from 2 to 3 records.
We saw the DMA engine could provide a throughput boost
of up to 55\% over copy-out. If range scans return even just 16 records per query, the
benefits of zero-copy purely from the perspective of transmission throughput is almost eliminated.
% computePctTputImprovementForZeroCopy(loadMerged())
% [1] "128 B item pct improvement in tput for zero-copy"
%  [1]   8.105552  10.928705  14.097586  18.910805  28.721654  43.346138  51.993156  44.439383  55.446378
% [10]  43.414034  35.736397  26.654478  23.842016  20.787579  18.376618  -5.310453  -2.382863  -4.469114
% [19]  -5.565349  -6.927329  -6.270677  -7.919818  -8.681226  -9.356702 -11.269678 -11.943183 -12.265026
% [28] -13.209408 -13.396745 -13.930867 -14.057826 -12.286999
% [1] "Max 55.4463782269341"
% [1] "1024 B item pct improvement in tput for zero-copy"
%  [1] 53.841302 86.064417 55.299614 40.553317 37.989062 33.750689 30.841420 23.594658 21.917508 19.037890
% [11] 18.186771 15.621109 14.475694 12.246097 11.008283  7.318041  7.586871  7.262891  7.572221  7.396499
% [21]  7.427524  7.109614  7.039692  6.980079  6.610857  5.824279  5.979297  5.631181  5.694890  5.299445
% [31]  5.746029  5.610679
% [1] "Max 86.0644167799178"

Next, we consider 128~B records. The decreased access latency of
in-memory databases makes them well-suited to smaller, finer-grained records
than were previously common. One expectation is that this will drive databases
toward more aggressively normalized layouts with small records. This
seems to be increasingly the case as records of a few hundreds bytes or less
are now typical~\cite{fb-memcache,fb-workload}.

%Copy Out 128 B - 8chunks(1024B) - 2400 MB/s
%Copy Out 128 B - max 32 chunks - 3885MB/s
%Copy out 1024B  -max 32chunks - 5507MB/s
%ZeroCopy 128 B - chunks-MB/s, 1-526,7-3430,8-3466,9-3450,10-3417,
% 11-3431,12-3375,13-3437,14-3493,15-3532,16-2908, 30-3314,31-3304,32-3408
%ZeroCopy 1024 B - 1-3841, 2-5861, 32-5816

Figure~\ref{fig:zero-copy-tput} shows that for small 128~B records, the NIC DMA
engine provides little throughput benefit if you were only returning a few records 
at a time. Our NIC is limited to gather lists
of 32~entries, which is insufficient to saturate the network with such a small
record size. Transmission peaks at 3.4~GB/s. In fact, Zero Copy performs best when
we transmit between 8-15 records or 1 to 2~KB per transmission above which the transmit performance 
tapers off. This comparison is evident and Zero Copy gives around 85\% improvement over Copy Out while dealing 
with transmissions around 1~KB or 8 records. We observe a great dip when we get to 16 records and it gradually moves up as we 
increase the transmission sizes further but never gets back to the peak transmission. This makes 
for an interesting observation for a database designer who assumes smaller record sizes and larger 
transmissions. We dig deeper into this anomaly and find evidence as to why this happens with the 
help of measuring traffic induced in the Memory controller due to LLC misses because of DDIO and 
PCIe traffic in Chapter~\ref{chap:impact}. Owing to this abberation, copying 128~B records 
on-the-fly can significantly outperform zero-copy transmission when there are 
enough results (more than 16 records in our NIC which is capable of transmitting 32 at once) to group per transmission.
In fact, copy-out can saturate the network with small records, and it
performs somewhat identically to copy-out with larger 1024~B records. We discovered this anomaly 
with the use of thread local storage in the benchmarks and more fine-grained locking 
in our own implementation. The measurements in this thesis only discusses the 
measurements obtained after these optimisations.

\subsubsection{Zero-copy Savings}
\label{sec:overhead}

\begin{figure}[t]
\includegraphics{fig-overheads.pdf}
\caption{Breakdown of absolute CPU overheads for transmission.}
\label{fig:overheads}
\end{figure}

While we just showed in the above section that judicious use of Zero Copy results in  enhanced transmission performance
, the goal of zero-copy DMA is to mitigate the role of server-side CPU. Figure~\ref{fig:overheads} breaks down
CPU time for all the scenarios; small and large records using Zero Copy and Copy-Out. We can observe that in the case 
of zero copy, most of the server CPU time is spent idling waiting for the NIC 
to complete transmissions at smaller record sizes. We can also see that for 128~B records,
smaller transmissions take up more fraction of CPU owing to the overhead of more descriptors
per transmission. It is also obvious that that zero-copy always reduces 
the CPU load of the server in all record sizes, and, as expected, there is a bigger
benefit for larger record sizes. 
With 1024~B records, the \memcpy  step of copy-out uses a maximum of 18\% of all available
CPU cycles. This is a pure overhead that we could eliminate by using Zero Copy.
While Zero Copy eleminates the \memcpy overhead, it adds an overhead of its own to create  transmit descriptors.
Each gather entry adds 16~B to the descriptor that is posted to the NIC.
These entries are considerable in size compared to
small records, and they are copied twice. The gather list
is first staged on the thread's stack and passed to the userlevel NIC driver. Next,
the driver makes a posted PCIe write by copying the descriptor (including the
gather entry) into a memory-mapped device buffer.

The memory bandwidth savings for zero-copy are more substantial.
Figure~\ref{fig:zero-copy-tput} shows that copy-out transmit performance nearly
matches zero-copy (5.6~GB/s versus 5.8~GB/s) for larger records at larger transmissions. Copy-out introduces exactly one
extra copy of the data, and \memcpy~reads each cache line once and writes it
once. So, copy-out is expected to increase memory bandwidth consumption
by 2$\times$ the transmit rate of the NIC or 11.2~GB/s in the worst case.  This
accounts for about 45\% of the available memory bandwidth for the server that we used. Whether
using zero-copy or copy-out, the NIC must copy data from main memory, across the PCIe bus, to its own buffers, which
could end up using another 6~GB/s of memory bandwidth. With these estimates in mind, we extensively profiled
 our benchmark with Intel's PMU module and the details of that is given in Chapter~\ref{chap:impact}.

% Need to know how much memory bw \memcpy/  actually burns.
% Based on pcm-memory.x and the membench tool I stuff in the ibv-bench
% directory \memcpy/  does one read and one write for each cache line it copies,
% probably at least when the copies are larger enough.
% Based on Erik's work it looks like row-buffers are 8 KB! Wow, this might
% actually impact our story in terms of energy efficiency!

% computeBestCyclesPerRecordImprovement(loadMerged())
% [1] "1773889090.13867 624551729.013333 2961703308.21867 148009809.173333"
% [1] "Pct overhead reduction of nocp versus cp for"
% [1] "128 B records"
% [1] 64.79195
% [1] "1024 B records"
% [1] 95.00254
% > 


\begin{figure}[t]
\includegraphics{fig-cycles.pdf}
\caption{Cycles per transmitted byte for large and small records with Zero Copy and Copy Out. Note the log-scale axes.}
\label{fig:cycles}
\end{figure}

The results break down where the CPU and memory bandwidth savings come from,
but not all configurations result in the same transmit performance. For
example, Figure~\ref{fig:zero-copy-tput} shows that when transmitting
128~B~records, copy-out gets up to 15\% better throughput than zero-copy while transmitting
32~records at once. As a result, minimizing CPU overhead can come at the expense of transmit
performance.  The real CPU efficiency of the server in transmission is shown in
Figure~\ref{fig:cycles}. The figure shows how many cycles of work the CPU needs to do
for every byte transmitted in each of the configurations, which reveals two key things.
First, it shows that, though the absolute savings in total CPU cycles is small
for zero-copy, it does reduce CPU overhead due to transmission by up to 95\% for 
larger records and around 64\% for smaller records. For smaller records, small transmission
results in more CPU usage (till around 3~records) owing to the overhead of descriptors.


% LMBench 3.0 Memory Results from our Machines
% *Local* Communication bandwidths in MB/s - bigger is better
% -----------------------------------------------------------------------------
%  Host                OS  Pipe AF    TCP  File   Mmap  Bcopy  Bcopy  Mem   Mem
%                               UNIX      reread reread (libc) (hand) read write
% --------- ------------- ---- ---- ---- ------ ------ ------ ------ ---- -----
% node-0.st Linux 2.6.32-                              4813.3 5606.8 8349 6797.
%

% Copy 512 MB between two regions of a single hugepage 100 times.
% One process at a time:
% [stutsman@node-0 ~]$ ./membench
% CPU Secs 12.023564
% bw 4.158501 GB/cpuS
%
% 8 processes at a time:
% [stutsman@node-0 ~]$ CPU Secs 49.799340
% bw 1.004029 GB/cpuS
% CPU Secs 49.842697
% bw 1.003156 GB/cpuS
% CPU Secs 49.905588
% bw 1.001892 GB/cpuS
% CPU Secs 50.420696
% bw 0.991656 GB/cpuS
% CPU Secs 50.585852
% bw 0.988419 GB/cpuS
% CPU Secs 50.699039
% bw 0.986212 GB/cpuS
% CPU Secs 50.757823
% bw 0.985070 GB/cpuS
% CPU Secs 50.850676
% bw 0.983271 GB/cpuS

% > sum(d[d$chunkSize == 128 & d$chunksPerMessage == 64,]$mbs)
% [1] 4438.671


