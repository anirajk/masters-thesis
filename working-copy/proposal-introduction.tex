%%% -*-LaTeX-*-

\chapter{Introduction}
In-memory databases sparked a new wave of systems research in the 
networking and database communities. For a long time while developing storage
systems, researchers didn't diverge from the  tried and tested model for
locality of reference. Conventional ``secondary storage" devices
such as disks were the final resting place of evergrowing amounts of managed data. With the 
enhanced requirements of scale by the popularization of Web 2.0 in the early 2000s, 
database designers created a caching layer of RAM in front of disks
to combat abysmal disk access times that was dragging performance at scale. Memcached~\cite{memcached-orig}
is one such phenomenal system that stood the test of time and it's further development
enabled it to operate at a scale of billions of operations per second on trillions of items~\cite{nishtala2013scaling}.

Another orthogonal area of research was to completely store all data in main memory.
This was made possible in the late 2000s by the evolution of DRAM technologies and declining cost of main memory.
Questions about durability and cost of replication 
were responded by means of distributed recovery~\cite{ongaro2011fast}. 
A more recent development was in the acceptance of high throughput and low latency networks. 
Pioneered by the high performance computing community, these advanced networking fabrics 
such as infiniband~\cite{pfister2001introduction} became more commonplace in 
data center networks. These developments have led to a deluge of storage systems research~\cite{mmdbmstutorial} ,
exploring multiple verticals. Among the particularly notable ones are RAMCloud~\cite{ramcloud},
MICA~\cite{mica}, FaRM~\cite{farm}, HyPer~\cite{hyper}, H-Store~\cite{hstore} 
and HERD~\cite{herd}. While each of these systems have been driven by different dimensions 
of research, they all share the same concerns of large scale in-memory storage systems 
operating on high speed networks.


With the increasing popularity of In-memory databases, database design landscape changed significantly.
Spinning disks which were the primary bottleneck in performance were suddenly 
of little to no concern. Storage systems started to throw bigger and bigger cache layers to mitigate 
disk-greater-than-memory~\cite{couchbase} effects. In-memory databases effectively eleminate disk I/O
as a concern and are capable of throughputs of the order of millions of operations per second
at microsecond latencies. While this might tempt us to believe that database I/O is completely
solved, we argue that there is a new device that dictates the overall performance of these 
distributed systems: The Network Interface Card (NIC).


Modern network cards have grown incredibly complex, partly in response to
demands for high throughput and low latency. One key advancement these NICs made 
is the presence of some form of kernel bypass where an on-device DMA engine from the network card 
can access pre-mapped memory in a system without invoking the CPU. This led to the Remote 
Direct Memory Access (RDMA)~\cite{rdmapatent} paradigm. The modern NIC with it's complex architecture
makes understanding it's characteristics difficult, and it makes designing software to take advantage
of them even harder. Database designers encounter a number of trade-offs when trying to use the NIC effectively.
We specifically looked at three factors that make it especially challenging to design NIC-friendly
database engines:
\begin{itemize}
  \item {\em Zero-copy} DMA engines reduce server load for transferring large data
    blocks, but large, static blocks are uncommon for in-memory databases,
    where records can be small and update rates can be high.
  \item The performance gains from zero-copy DMA do not generalize to
      {\em finer-grained objects} for two reasons: (1) transmit
      descriptors grow linearly in the number of records;
      (2) NIC hardware limits descriptor
      length, limiting speed for small records.
      Despite this, we find that zero-copy
      can make sense for small objects under certain conditions, such as
      small ``add-ons'' to larger transmissions.
   \item Zero-copy introduces complications for \emph{locking and object
      lifetime}, as objects must remain in memory and must be
      unchanged for the life of the DMA operation (which includes transmission
      time).
\end{itemize}


These factors make advanced NIC DMA difficult to use effectively. We set out to explore data 
transmission purely from the perspective of a modern NIC. We profiled a cluster of machines with
Infiniband ConnectX\textregistered-3, a modern kernel-bypass capable NIC
with RDMA capabilities, with a specific attention to large transfers and query responses.
We proposed our findings and made some concrete suggestions around how NICs might
influence data layout and concurrency control in a modern in-memory 
databases~\cite{KesavanRicciStutsman:IMDM16}. 

Our primary results gave us ideas about how the Zero Copy paradigm will perform against
the traditional approach of transmitting via larger buffers. These also pointed 
to how the use of thread local storage and effective caching such as the 
Intel\textregistered Data Direct I/O~\cite{ddio} could boost performance and 
efficiency of copying data. 

Yet another burden of using these newer storage systems are how they would provide effective 
reconfiguration. Our results from the benchmark led us to believe in the importance of 
hardware considerations in data migration. Investment in costly hardware and aggressive
service level agreements(SLAs) make the task of effective reconfiguration a significant research challenge. 
This is also one of the key hurdles to overcome before mainstream acceptance of these systems.
We explored the state of the art reconfiguration systems such as Squall~\cite{squall} which 
introduced live reconfiguration by means of reactivelly pulling data with numerous optimisations.
While a system like this creates little impact on the source in theory, it's performance improvements
are dependent on specific workloads and pre partitioning of data. We wanted to aim for a solution that defers 
partitioning decisions as much as possible while still exploring benefits of data locality in a complex
In-memory system optimised for low latency such as RAMCloud~\cite{ramcloud}.


RAMCloud has a unique perspective on data locality. While there are obvious benefits to locality
for building distributed transactions and operations such as multi-gets that are supported by 
RAMCloud, RAMCloud doesn't prohibitively couple flexibility of data placement with performance like other systems 
such as H-Store~\cite{hstore}. Since it's design is driven by low latency access, there is less incentive
to design for locality. At the same time, improvement such as multiple secondary indices~\cite{slik} and need for reconfiguration 
makes collocation of data a nice-to-have to say the least. 

We evaluate the current state of the art approaches to migration on a log structured,
in-memory database comparing it with other methods. We lay out some experiments
that points out issues with current approaches and motivates the necessity 
for a fast and efficient migration protocol. These will explore the themes of data locality,
source impact and the effect of numerous workloads which include scaling up and down the cluster, splitting
and migrating hot index tables, the impact on locality by the distributed recovery which sprays data across the 
cluster and responding to changes in access patterns taking into account the multi dimensional resource contentions
that could occur in a practical system.

We will also propose the outline for an effective migration protocol that leverages
the log structured memory of RAMCloud as well as the distributed crash recovery mechanism
to provide fast and jitter free data movement in a cluster for reconfiguration and improved 
performance for range queries and distributed transactions. We will try to do so while
defering partitioning decisions as much as possible.

\section{Contributions}

\blockquote{\textbf{Scale-out In-Memory stores are optimised for small requests
under tight SLAs, and bulk data movement, for rebalancing and range queries, interfere;
We hypothesize that carefully leveraging data layout and advancements in modern NICs
will yield to gains in performance and efficiency for large transfers in these systems
without disrupting their primary obligation}}



\begin{itemize}
\item{\textbf{Evaluation of Impact of Modern NICs on data layout}}: We profiled a modern NIC under various 
scenarios and data transfer mechanisms exploring the scatter gather DMA engine of the NIC in contrast
with the conventional copy out approach of pre aggregating data in a buffer and sending data from the buffer.
We even found a few arguments against the blind use of the new kernel by pass approach considering the overhead
of concurrency management and difficulty of implementation. We evaluated various scenarios on top of a
microbenchmark to measure transmit performance of responses varying message sizes and number of records
transmitted.
\item{\textbf{Design of client assisted responses for better transmit performance}}: While evaluating the results,
we saw that there are certain structures and circumstances that can uniquely take advantage of the hardware limited 
scatter gather entries of a modern NIC. One such exploration was around a lock-free, no update in place 
indexing data structure such as the Bw-Tree~\cite{bw-tree}. We propose a design with client assisted 
validation techniques in the light of these observations that essentially proposes sending more data than 
needed might in fact boost overall throughput of a system.
\item{\textbf{Experiments to motivate fast and efficient data transfer}}: We will lay out experiments that will 
show how even the state of the art systems for cluster reconfiguration becomes suboptimal with it's locality
constraints and overall slowness. We will layout experiments that will call for fast and efficient migration protocol
especially suited for an In-memory database system with a decentralized log structured data model such as RAMCloud.
This will involve analysing the existing protocol in depth and zeroing in on the fractions of time spent on migration.
\item{\textbf{Preliminary design for a migration protocol}}: We will learn the most important bottlenecks in the current design
that makes it infeasible for fast and efficient migration and in turn propose the basics of a migration protocol that 
will effectively eleminiate these bottlenecks and pave the way forward for fast and efficient transfer of huge chunks of data.
\end{itemize}

