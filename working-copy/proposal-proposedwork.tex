%%% -*-LaTeX-*-

\chapter{Proposed Work}
\label{chap:proposed-work}
This is a forward looking section listing out a few experiments that will
strengthen the thesis and pave way for a fast and efficient migration protocol.
Table~\ref{tbl:timeline} shows the expected amount of work that needs to be done 
in the coming months to conclusively gather evidences that support the thesis. 
This will also serve as the schedule guideline for the rest of my program.

\section{DDIO}
We need to find conclusive evidence about the impact of synchronization and 
advanced caching mechanism on our experiments. The synchronisation primitives 
that are used in the microbenchmarks should not have a driving effect to the 
conclusions we make from them. This includes fully investigating the impact
of DDIO~\cite{ddio} as well as coming up with uncore and off-core performance
measurements to pinpoint the areas of contention while transmitting huge blocks
of data. So far, we have analysed the impacts on memory bandwidth by the traditional
copy out approach without considering the changes in access patterns and the effects
of Direct Caching~\cite{dca}. We should investiaate the last level cache characteristics 
and other performance events as prescribed in Intel\textregistered 's software development
manual in order to gather accurate performance measurements using the perf tool.

\section{Data Migration}
We also need to motivate the need for a fast and efficient migration protocol 
by virtue of uncovering bottlenecks in the current state of the art. This would entail
setting up the experiments showing data migration of RAMCloud using the latest 
available protocol and analysing and finding pain points that hamper performance.
After careful analysis, this needs to translate to performance benefits in the absense
of thus found bottlenecks. 

We will analyse the upperbound of each measurements with the help
of previous benchmarks showing the maximum transmission rate achieveable under 
a cluster of modern hardware. This will show how far we are away from 
operating at line rate and provide concrete evidence of limitations present in 
the current approach. Also, there needs to be another experiment which measures
the impact of indexes in migration for change in load and skew on indexed reads
in a system such as RAMCloud. We are hopeful that these will provide interesting
insights on data locality.

\section{Timeline}
\begin{table}[t]
\def\arraystretch{1.25}%  1 is the default, change whatever you need
\begin{tabular}{lll}
\toprule
\textbf{DDIO} & & 4 weeks \\
\midrule
i & Calculate Impact on memory bandwidth & 2 weeks \\
ii & Get uncore/off-core measurements  & 2 weeks \\
\midrule
\textbf{Migration} & & 5 weeks\\
\midrule
i & Setup experiment to motivate index migration & 2 weeks \\
ii & Analyse current protocol to get bottlenecks & 3 weeks \\
\bottomrule
\end{tabular}
\vspace{0.25eX}
\caption{Expected Timeline for pending experiments.}
\label{tbl:timeline}
\end{table}



\begin{itemize}
\item{\textbf{Till Fall 2016}:} We have evaluated how modern NIC impacts data layout 
and concurrency by developing a microbenchmark and measuring throughput and cpu 
impact while transmitting huge amounts of data. We compared how scatter gather DMA 
and the traditional copy impacts throughput varying record sizes and layout. 

\item{\textbf{Fall 2016 - Spring 2017}:} We have to explore the impacts of Intel\textregistered 
DDIO~\cite{ddio} technology which makes the last level cache the primary source and 
destination memory for the NICs. We need to do this in the context of thread local 
buffers to conclude their effect on modern NICs. We also need to set up a few experiments
that will motivate fast and efficient migration protocol. 

This would involve  setting up migrations involving huge amount of data and measuring it's impact on overall throughput
and experiments where it can be proved that SLAs of the system are violated as an effect of migration.
We would examine the performance of current migration protocol and come up with in depth measurements
that will help analyse the protocol both conceptually and from the perspective of implementation in RAMCloud.
The primary set of experiments that need to be done will migrate data and measure throughput and time 
spent on each individual step in migration and finding ways to mitigate the bottlenecks to analyse
the impact as part of a big picture.

In addition to the the primary experiment, there could also be other experiments which will pave 
the way for a new protocol. One such possible experiment is to split and migrate a hot index table which has it's 
records spanning multiple files. Other possible experiments include measuring the 
impact of distributed recovery on locality, scaling a cluster up and down and measuring
difference in performance characteristics, shifting $\theta$ on a workload which follows 
Zipfian distribution etc.
\end{itemize}


%\section{Evaluation}
%%%% -*-LaTeX-*-
%Evaluation has been folded into other chapters as per Ryan's suggestion.