%%% -*-LaTeX-*-

\chapter{Proposed Work}

This is a forward looking section listing out a few experiments that will
strengthen the thesis and pave way for a fast and efficient migration protocol.
We need to find conclusive evidence about the impact of synchronization and 
advanced caching mechanism on our experiments. The synchronisation primitives 
that are used in the microbenchmarks should not have a driving effect to the 
conclusions we make from them. Another aspect that needs further investigation 
is the access patterns. For our initial experiments, we completely randomised the data which is often 
not the case in a practical system. We need to thoroughly investigate the effects
of caching on transmit performance especially since newer NICs have the last level
cache as their primary source and destination for data transfer.

\section{Timeline}

\begin{itemize}
\item{\textbf{Till Fall 2016}:} We have evaluated how modern NIC impacts data layout 
and concurrency by developing a microbenchmark and measuring throughput and cpu 
impact while transmitting huge amounts of data. We compared how scatter gather DMA 
and the traditional copy impacts throughput varying record sizes and layout. 

\item{\textbf{Fall 2016 - Spring 2017}:} We have to explore the impacts of Intel\textregistered 
DDIO~\cite{ddio} technology which makes the last level cache the primary source and 
destination memory for the NICs. We need to do this in the context of thread local 
buffers to conclude the effect of modern NICs. We also need to set up a few experiments
that will motivate fast and efficient migration protocol. This would involve 
setting up migrations involving huge amount of data and measuring it's impact on overall throughput
and experiments where it can be proved that SLAs of the system are violated as an effect of migration.
One such possible experiment is to split and migrate a hot index table which has it's 
records spanning multiple files. Other possible experiments include measuring the 
impact of distributed recovery on locality, scaling a cluster up and down and measuring
difference in performance characteristics, shifting $\theta$ on a workload which follows 
Zipfian distribution etc.
\end{itemize}