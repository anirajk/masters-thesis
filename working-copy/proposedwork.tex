%%% -*-LaTeX-*-

\chapter{Proposed Work}
\label{chap:proposed-work}
This is a forward looking section listing out a few experiments that will
strengthen the thesis and pave way for a fast and efficient migration protocol.
Table~\ref{tbl:timeline-pending} shows the expected amount of work that needs to be done 
in the coming months to conclusively gather evidences that support the thesis. 
This will also serve as the schedule guideline for the rest of my program.


\section{DDIO}
Modern NICs can directly transmit or receive data from CPU's last level cache.
We haven’t yet fully evaluated this effect, and it could have significant impact on
the design of efficient migration. For example, the overhead of \memcpy and CPU
cache misses might at the source of migration may be offset by reduced transmission
costs. We plan to repeat our ibv-bench experiments from the IMDM paper
to assess whether we can exploit this effect for migration.


\textbf{Analyse effects of Direct Caching on memory bandwidth}: We need to find conclusive evidence about the impact of synchronization and 
advanced caching mechanism on our experiments. This would involve fully 
investigating the impact of DDIO~\cite{ddio} on memory bandwidth and figuring out 
a way to toggle the effect to compare before and after effects of Direct Caching.

\textbf{Get uncore and offcore performance measurements from the microbenchmark}: 
We need to re-run the ibv-benchmark to come sup with uncore and off-core performance
measurements to pinpoint the areas of contention while transmitting huge blocks
of data. So far, we have analysed the impacts on memory bandwidth by the traditional
copy out approach without considering the changes in access patterns and the effects
of Direct Caching~\cite{dca}. We should investiaate the last level cache characteristics 
and other performance events as prescribed in Intel\textregistered 's software development
manual in order to gather accurate performance measurements using the perf tool.

\textbf{Evaluating the effect of thread local storage}: The synchronisation primitives 
that are used in the microbenchmarks should not have a driving effect to the 
conclusions we make from them. We need to get new numbers for our measurements with
thread local buffers to see the impact.

\textbf{Recalibrating measurements for Connect-IB\textregistered}: Once we get the 
measurements accounting for the effects of Direct Caching and thread local buffers,
we need to re-evaluate Connect-IB\textregistered (newer) cards and compare them 
with the Connect-X 3\textregistered cards to evaluate the difference in 
throughput that can be achieved by Zero Copy by these devices in the light of new findings.



\section{Data Migration}
We are currently profiling RAMCloud’s existing migration mechanism, which seems to max out
at around 100MB/s, which is about two orders of magnitude away from the NIC's full line rate.
The key deliverable is a piece-wise assessment of the bottlenecks of the existing migration
protocol, obtained by progressively removing bottlenecks and measuring gains. Since 
indexes and data tables in RAMCloud are handled differently for migration, we need
to assess them both separately.


\textbf{Analysing bottlenecks in RAMCloud migration protocol}: We need to motivate the need for a fast and efficient migration protocol 
by virtue of uncovering bottlenecks in the current state of the art. This would entail
setting up the experiments showing data migration of RAMCloud using the latest 
available protocol and analysing and finding pain points that hamper performance.
After the analysis, this needs to translate to performance benefits in the absense
of disovered bottlenecks.  We will analyse the upperbound of each measurements with the help
of previous benchmarks showing the maximum transmission rate achieveable under 
a cluster of modern hardware. This will show how far we are away from 
operating at line rate and provide concrete evidence of limitations present in 
the current approach. Also, there needs to be another experiment which measures
the impact of indexes in migration for change in load and skew on indexed reads
in a system such as RAMCloud. We are hopeful that these will provide interesting
insights on data locality.

\textbf{Come up with guidelines for new protocol}: Based on all our measurements,
we need to come up with guidelines for a new migration protocol that will be efficient
and jitter free while keeping up with the tight SLAs of RAMCloud. This new protocol 
will be at least a 1000x faster than the state of the art migration mechanism 
for in-memory stores.
\begin{table}[H]
\def\arraystretch{1}%  1 is the default, change whatever you need
\begin{tabular}{|l|l|l|l|}
\hline
{Task 1} & \begin{tabular}{@{}l@{}} Develop a microbenchmark to measure\\
performance of infiniband verbs \end{tabular} & Feb 2016 & Done\\
{Task 2} & \begin{tabular}{@{}l@{}} Analyze performance of RDMA\\
 verbs over ibverbs library \end{tabular} & Mar 2016 & Done\\
{Task 3} & \begin{tabular}{@{}l@{}} Measure Transmit performance\\
 and CPU cycles for Zero Copy and Copy Out \end{tabular}& Apr 2016 & Done\\
{Task 4} & Compare Peformance of Delta Records  & May 2016 & Done\\
{Task 5} & Write up and submit results to IMDM-2016  & Jul 2016 & Done\\
{Task 6} & \begin{tabular}{@{}l@{}} Follow up measurements on \\
Connect-IB\textregistered and hugepages\end{tabular} & Aug 2016 & Done\\
{Task 7} & \begin{tabular}{@{}l@{}} Researching state of the art\\ 
protocols for in-memory migration\end{tabular} & Oct 2016 & Done\\
{Task 8} & Write up Abstract & Oct 2016 & Done\\
{Task 9} & Finish Writing Chapter on Introduction & Nov 2016 & Done\\

\hline
\textbf{Task 10} & \begin{tabular}{@{}l@{}} Analyse effects of DDIO\textregistered \\
 on memory bandwidth \end{tabular}& Nov 2016 & In Progress\\
\textbf{Task 11} & Finish the chapter on modern NICs & Nov 2016 & Done\\
\textbf{Task 12} & Defend the thesis proposal & Nov 2016 & In Progress\\
\textbf{Task 13} & \begin{tabular}{@{}l@{}} Get uncore and offcore performance \\
measurements from the microbenchmark\end{tabular} & Dec 2016 & TBD\\
\textbf{Task 14} & \begin{tabular}{@{}l@{}}Analysing bottlenecks in \\
RAMCloud migration protocol\end{tabular} & Dec 2016 & TBD\\
\hline
\textbf{Task 15} & Finish writing migration chapter & Jan 2017 & TBD\\
\textbf{Task 16} & Come up with guidelines for new protocol & Jan 2017 & TBD\\
\textbf{Task 17} & Finish writing Conclusion & Jan 2017 & TBD\\
\textbf{Task 18} & Final thesis defense & Feb 2017 & TBD\\
\hline
\end{tabular}
\vspace{0.25eX}
\captionsetup{justification=centering}
\caption{Thesis Timeline. Pending tasks boldfaced.}
\label{tbl:timeline-pending}
\end{table}



%\section{Evaluation}
%%%% -*-LaTeX-*-
%Evaluation has been folded into other chapters as per Ryan's suggestion.
